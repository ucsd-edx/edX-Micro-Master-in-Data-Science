\documentclass[10pt]{article}

\newcommand{\littleblank}{\rule{.5in}{.01in}}
\newcommand{\blank}{\rule{1in}{.01in}}
\newcommand{\bigblank}{\rule{2in}{.01in}}
\newcommand{\bigbigblank}{\rule{3in}{.01in}}
\newcommand{\CUT}[0]{}
\usepackage{fullpage}
\usepackage{graphicx}
\usepackage{amssymb}
\usepackage{fancyvrb}
\usepackage{fancyhdr}
\usepackage{float}
\usepackage{enumitem}
\usepackage{amsmath}
\usepackage{caption}
\usepackage{subfig} 
\usepackage{bm}

\pagestyle{fancy}
\renewcommand{\headrulewidth}{0pt}
\rhead{PID: \underline{\hspace{1in}}}

\vspace{.1in}
\begin{document}

\setlength\parindent{0pt}
\thispagestyle{empty}

{\textbf \Large Quiz 3} \hfill CSE 255, Spring 2017
\\

\vspace{.1in}

Name: \underline{\hspace{3in}}
\\

PID: \underline{\hspace{3.15in}}

\vspace{.1in}

{\small \setlength\parindent{20pt}This is the fourth quiz of CSE255/DSE230

On your desk you should have only the exam paper and writing tools.
No hats or hoods allowed (unless religious items).
There are 4 questions in this exam, totalling 100 points.

You have 10 minutes to complete the exam.

Start by writing your name and PID on this page.

Good Luck!}\\
\underline{\hspace{6in}}

\newcommand{\uu}{\vec{u}}
\newcommand{\xx}{\vec{x}}
\newcommand{\yy}{\vec{y}}
\newcommand{\vmu}{\vec{\mu}}
~\\
~\\
\vspace{.1in}
\noindent
    {\bf Setup}
    \begin{itemize}
    \item $T=\{\xx_1,\xx_2,\xx_n\}$ is a set of vectors in $R^d$.
    \item The mean of $T$ is $\vmu = \frac{1}{n} \sum_{i=1}^n\xx_i$
    \item $C$ is the covariance matrix of $T$. The eigenvectors of $C$
      are $\uu_1,\ldots,\uu_d$ and the corresponding eigen-values are
      $\lambda_1 > \lambda_2 > \cdots > \lambda_d$
    \end{itemize}
    The answers to the questions below should consist of the vectors
    and scalars defined above (and potentially constant numbers).\\
    ~\\
    ~\\
\noindent
    {\bf I. (25 points):} \\
    Write an expression for $\frac{1}{n} \sum_{i=1}^n (\xx_i - \vmu)\cdot(\xx_i -
    \vmu)$ \\
~\\
~\\
\underline{\hspace{6in}}\\
\vspace{.1in}
\noindent
    {\bf II. (25 points):} \\
    Write an expression for the approximation of $\yy \in R^d$ using
    the first two eigen-vectors of $C$ \begin{small} (note: $\hat{y}^2$ denotes the
    approximation vector using the first two eigen-vectors, it is {\bf
      not} the square of $\hat{y}$.) \end{small} \\
~\\
~\\
\underline{$\hat{y}^2=$ \hspace{6in}}\\
~\\
\vspace{.1in}
\noindent
    {\bf III. (25 points):} \\
    Write an expression for the {\em residual} of $\hat{y}^2$
~\\
~\\
    \underline{$\hat{r}^2=$ \hspace{6in}}\\
~\\
\vspace{.1in}
\noindent
    {\bf IV. (25 points):} \\ Denote by $\hat{x}_i^2$ the
    approximation of $\xx_i$ using the first two eigenvectors. Write
    an expression for\\ $\frac{1}{n} \sum_{i=1}^n (\hat{x}_i^2 -
    \vmu)\cdot(\hat{x}_i^2  - \vmu)$
    ~\\
    ~\\
    \underline{\hspace{6in}}\\

\end{document}   




 
