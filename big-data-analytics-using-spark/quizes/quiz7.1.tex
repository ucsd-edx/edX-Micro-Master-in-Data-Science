\documentclass[10pt]{article}

\newcommand{\littleblank}{\rule{.5in}{.01in}}
\newcommand{\blank}{\rule{1in}{.01in}}
\newcommand{\bigblank}{\rule{2in}{.01in}}
\newcommand{\bigbigblank}{\rule{3in}{.01in}}
\newcommand{\CUT}[0]{}
\usepackage{fullpage}
\usepackage{graphicx}
\usepackage{amssymb}
\usepackage{fancyvrb}
\usepackage{fancyhdr}
\usepackage{float}
\usepackage{enumitem}
\usepackage{amsmath}
\usepackage{caption}
\usepackage{subfig} 
\usepackage{bm}

\pagestyle{fancy}
\renewcommand{\headrulewidth}{0pt}
\rhead{PID: \underline{\hspace{1in}}}

\vspace{.1in}
\begin{document}

\setlength\parindent{0pt}
\thispagestyle{empty}

{\textbf \Large Quiz 7} \hfill CSE 255, Spring 2017
\\

\vspace{.1in}

Name: \underline{\hspace{3in}}
\\

PID: \underline{\hspace{3.15in}}

\vspace{.1in}

{\small \setlength\parindent{20pt}This is the 7'th quiz of CSE255/DSE230

On your desk you should have only the exam paper and writing tools.
No hats or hoods allowed (unless religious items).
There is one question in this quiz.

You have 15 minutes to complete the exam.

Start by writing your name and PID on this page.

Good Luck!}\\
\underline{\hspace{6in}}

\newcommand{\uu}{\vec{u}}
\newcommand{\xx}{\vec{x}}
\newcommand{\yy}{\vec{y}}
\newcommand{\vmu}{\vec{\mu}}
~\\
~\\
\noindent
    {\bf Problem I}\\
   Suppose examples are generated in the following way. The input
   feature $x$ is a scalar drawn uniformly from the range
   $[0,1]$. The label is one of ${+1,-1}$ and is generated
   according to the conditional probability
   $$P(y=+1 | x) = \left[ax +b\right]$$ where $a,b>0$ are unknown
   parameters and the square
   brackets indicate restricting the variable between $0$ and $1$
   $$[x] = \max(0,\min(x,1))$$. You are given a training set,
   generated IID according to the given distribution:
   $(x_1,y_1),(x,2,y_2),\ldots,(x_m,y_m)$

   \begin{itemize}
   \item Write an expression for the Bayes optimal rule.
     \\~\\
     \underline{\hspace{6in}}
   \item Write an expression for the Bayes error (the error of the
     Bayes Decision rule.
     \\~\\
     \underline{\hspace{6in}}
   \item What would you try to estimate if you were using 
     generative models?
     \\~\\
     \underline{\hspace{6in}}
   \item What set of rules should you consider if you were using
     generative models?
     \\~\\
     \underline{\hspace{6in}}
   \item What would be the optimal discriminative rule?
     \\~\\
     \underline{\hspace{6in}}
   \end{itemize}

    ~\\
    ~\\

\noindent
    {\bf Problem II (50 points)}\\
    Mark the correct statements regarding boosting and margins:
    \begin{itemize}
      \item Once boosting achieves zero error on the training
        error, any additional training is likely to increase the test
        error.
        \item Maximizing the margin means finding a classifier such
          that all training examples have a large positive margin.
        \item Boosting can usually achieve zero training error.
        \item Boosting can usually achieve zero test error.
        \item Abstaining on examples with small margins reduces the
          test error.
          \item Small changes in the training set is unlikely to
            change the prediction on large margin examples.
      \end{itemize}
\end{document}   




 
